\documentclass[12pt]{extreport}
\usepackage[T2A]{fontenc}
\usepackage[utf8]{inputenc}      

\usepackage[english,russian]{babel} 


\usepackage{amsmath,amsfonts,amsthm,amssymb,amsbsy,amstext,amscd,amsxtra,multicol}
\usepackage{indentfirst}
\usepackage{verbatim}
\usepackage{tikz} 
\usetikzlibrary{automata,positioning}
\usepackage{multicol} 
\usepackage{graphicx}
\usepackage[colorlinks,urlcolor=blue]{hyperref}
\usepackage[stable]{footmisc}
\usepackage[portrait, top=3cm, bottom=1.5cm, left=3cm, right=2cm]{geometry}

\usepackage{fancyhdr}
\pagestyle{fancy}
\renewcommand{\headrulewidth}{0pt}
\lhead{\fontfamily{fca}\selectfont {Сергеенкова Лилия Александровна} }

\rhead{\fontfamily{fca}\selectfont Домашнее задание 2}

\cfoot{}

\usepackage{titlesec}
\titleformat{\section}[block]{\Large\bfseries\filcenter {\setcounter{problem}{0}}  }{}{1em}{}

                                                                    
\newcommand{\divisible}{\mathop{\raisebox{-2pt}{\vdots}}}           
\let\Om\Omega


\begin{document}

{\bf 1.} 

a) При решении уравнения вида $ax+by=c$ возможны две ситуации, когда $c$ делится на НОД$(a,b)$, тогда уравнение имеет решения в целых числах и когда нет -- тогда решений нет: при любых целых x и y число $ax + by$ делится на НОД$(a,b)$ и поэтому не может равняться числу $c$, которое на НОД$(a,b)$ не делится.

Алгоритмом Евклида найдем НОД$(a,b)$:

НОД$(238,385)$ = НОД$(385, 238)$ = НОД$(238,147)$ = НОД$(147,91)$ = НОД$(91,56)$ = НОД$(56,35)$ = НОД$(35,21)$ = НОД$(21,14)$ = НОД$(14,7)$ = НОД$(7,0)$ = 7.

Таким образом, НОД$(a,b) =7$. $\frac{133}{ 7}= 19$. Следовательно, уравнение имеет решения в целых числах.

Разделим обе части уравнения на НОД$(a,b) =7$:

$34x+55y=19$

Необходимо найти все решения данного уравнения, заметим, что, если $x^{'}$ и $y^{'}$ -- решение, тогда и $x = x^{'}+bt$ и $y = y^{'} - at$ -- решение, где $t \in \mathbb{Z}$:

$a(x^{'}+bt) + b(y^{'} - at) = ax^{'} + by^{'} = c$

Значит, необходимо найти для начала любое частное решение уравнения. 

Сделаем замену: $x = 19 x_0, y = 19 y_0$ , тогда:

$34x_0 + 55y_0=1$

Будем решать данное уравнение расширенным методом Евклида. 

$55x_1 + 34y_1=1$

$34x_2 + 21y_2=1$

$21x_3 + 13y_3=1$

$13x_4 + 8y_4=1$

$8x_5 + 5y_5=1$

$5x_6 + 3y_6=1$

$3x_7 + 2y_7=1$

$2x_8 + y_8=1$

$x_9=1$ $y_9=0$

Теперь нам необходимо подняться наверх и восстановить $x_0$ и $y_0$.

$ax+by=c$

$y = \frac{c-ax}{b} = c^{'}-a{'}x$

$c^{'}-a{'}x =bt$

$a^{'}x + bt = c{'}$

То есть, сделав замену: $b_1y_1+a_1x_1=c_1$, где $a_1=b, b_1=a \:mod \:b$

Таким образом, получим в общем виде:

\begin{equation*}
\begin{cases}
y_n = \frac{1-a_n y_n+1}{b_n} \\
x_n = y_{n+1}

\end{cases}
\end{equation*}

Тогда получим: 

$y_8 = \frac{1-2*0}{1}=1, x_8 = 0$

$y_7 = -1, x_7 = 1$

$y_6 = 2, x_6 = -1$

$y_5 = -3, x_5 = 2$

$y_4 = 5, x_4 = -3$

$y_3 = -8, x_3 = 5$

$y_2 = 13, x_2 = -8$

$y_1 = -21, x_1 = 13$

$y_0= 13, x_0 = -21$

Таким образом, общее решение, согласно формулам, написанным выше, получится: $x = 19*(-21) +55t, y = 19*13-34t$, то есть, $ \mathbf{x = -399 +55t, y=247-34t}$

\bigskip
б) $143x + 121y = 52$
Найдем НОД$(143, 121)$ = НОД$(121, 22)$ = НОД$(22, 11)$ = НОД$(11, 0)$ = 11

Разделим обе части уравнения на НОД$(a,b) =11$:

Однако 52 не делится на 11, это уравнение попадает под второй случай, описанный выше. Следовательно, решений нет. 


\bigskip
{\bf 2.}  

Построим цепочку: 

$7^{13} \:mod \:167 = ((7^6)^2*7) \:mod\: 167 = 81^2*7 \:mod \:167 = 2$

\hspace{16 mm} $\uparrow$

$7^6\: mod\: 167 = (7^3)^2\: mod \:167 = 81 \:mod \:167 = 81$

\hspace{16 mm} $\uparrow$


$7^3 \:mod \:167 = (7^2*7) \:mod \:167 = 49*7\: mod\: 167 = 9$

\hspace{16 mm} $\uparrow$


$7^2 \:mod\: 167 = (7*7)\: mod \:167 = 49$ 

\hspace{16 mm} $\uparrow$


$7 \:mod \:167 = 7$

Таким образом, ответ: ${\bf 7^{13} \:mod \:167 = 2}$


\bigskip
{\bf 3.}  

Для начала немного подробнее распишем алгоритм: 

$x = qy+r$. Ищем такие $q$ и $r$.

Если $x$ -- четно: $\frac{x}{2} = qy + r, x = 2qy + 2r$

Если $x$ -- нечетно: $\frac{x-1}{2} = qy + r, x = 2qy + 2r+1$

Если $r$ стало больше $y$ в ходе умножения на 2, то $r=r-y, q=q+1$

\bigskip

{\bf Корректность:}

Докажем корректность алгоритма по методу математической индукции. 

0. Для $\forall y \geq 1 , x = 0$ алгоритм работает верно

1. Для $\forall y \geq 1, \forall i < x $ алгоритм работатет верно по предположению, то есть, пара чисел $(q,r) = Divide([x/2],y)$ посчитана верно. 

2. При делении $x$ на 2 и округлении вниз мы просто отрезаем последний бит в двоичной записи числа, затем при умножении на 2 мы сдвигаем на один бит влево. Теперь все зависит от последнего бита $x$: если он равен 0, то все, если нет -- то надо к $r$ прибавить 1 и если $r$ стало больше $y$ в ходе умножения на 2, то $r=r-y, q=q+1$ (корректность этих математичесских формул описана выше). 

\bigskip
{\bf Сложность:}

На вход поступают два числа x и y. Длина входа: $log \:x + log\: y = O(n)$, n -- число битов. 

Если построить стек рекурсии, то вызовов будет $log \:x$, так как на каждом шаге аргумент $x$ делится пополам и возможное прибавление 1 не играет роли, так как $x$ округляется вниз. Таким образом, глубина рекурсии -- $log \:x = O(n)$

На каждом шаге рекурсии выполняется деление и умножение на 2 $x$ и умножение на 2 $y$-- это побитовый сдвиг влево ($O(1)$). Операции сложения в худшем случае, наверху рекурсии, стоят $O(n)$. (оценка $log \:x$ и $log \:y$). Таким образом, на каждом шаге рекурсии, глубина которой $O(n)$, выполняются операции, которые стоят $O(n)$. Следовательно, итоговая сложность -- ${\bf O(n^2)}$.

\bigskip
{\bf 4.}  

1. F(3,5):

$5 = 101 \rightarrow  X S S X$ 

$a = X S S X, y=1$

$a[1] == X: y =1*3 =3$

$a[2] \neq X: y =3*3 =9$

$a[3] \neq X: y =9*9 =81$

$a[4] == X: y =81*3 =243$

Следовательно, $F(3,5) = 243$

2. $F(x, m) = x^m$  -- возведение x в степень m.

3. Данный алгоритм -- подобие алгоритма быстрого возведения в степень для числа в двоичной записи. 

Для начала кратко опишем этот алгоритм. 

При переводе степени m в двоичный вид, мы имеем запись вида: 

$\overline{n_k n_{k-1} ... n_0}$, где $n_i =\{0,1\}$

Если $n_i = 1$, то текущий результат возводится в квадрат и затем умножается на x (число, которое мы возводим в степень). Если $ n_i = 0$, то текущий результат просто возводится в квадрат. То есть, если степень четна -- возводим в квадрат, если нет -- возводится в квадрат и затем умножается на x. Данный алгоритм был реализован на примере во втором номере. 

Заметим, что предложенный алгоритм делает похожие операции: просто 1 закодирована двумя символами, каждый из которых выполняет одно из действий: возведение в квадрат, умножение на x, то есть, в итоге получаем тот же результат. Для нуля, который закодирован одной буквой S: выполняется именно возведение в квадрат. 

В начале вычеркивается символ S, как незначащий 0 в двоичной записи числа, который появляется, если первая цифра в двоичной записи -- 1. Следовательно, данный алгоритм возводит число в степень, но в отличие от классического алгоритма для чисел в двоичном представлении, делает это для чисел в иной кодировке.

4. 
Пусть k -- длина показателя m в битах: $k = log \:m$. На каждом шаге по закодированной последовательности $a$ совершается одна операция, которая по условию стоит $O(1)$. Закодированная последовательность содержит не более, чем $2k-1$ символов. Следовательно, сложность алгоритма: $O(log \:m)$


\bigskip
{\bf 5.}

1.  $T_1(n) = T_1(n-1) + cn$ 

$T_1(n-1) = T_1(n-2) + c(n-1)$ 

То есть, $T_1(n) = T_1(n-2) + c(n-1) + cn$ и т.д.

Таким образом, заметим, что $T_1(n) = T_1(3) + 4c + 5c + ... + c(n-1) + cn = 1 + c(4+5+...+n) = 1+ c(\frac{n(n+1)}{2} - 1-2-3) = \bf{\Theta(n^2)}$ 

2. $T_2(n) = T_2(n-1) + 4T_2(n-3)$ 

Решим данное рекуррентное уравнение через характеристический многочлен. 

$x^3=x^2+4$

$x^3-x^2-4=0$

$x_1=2, x_2=\frac{1}{2} (-1 + i\sqrt7), x_3=\frac{1}{2} (1 + i\sqrt7)$

То есть общее решение будет: 

$T_2(n) = A(2)^n + B(1.4)^ncos(110.7n) + C(1.4)^nsin(110.7n)$ 

Следовательно, $T_2(n)=\Theta(2^n)$, то есть, ${\bf log \:T_2(n) = \Theta(n)}$

3. Из второго пункта ${\bf T_2(n)=\Theta(2^n)}$.




\end{document}